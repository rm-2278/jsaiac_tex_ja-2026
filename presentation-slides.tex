%!TEX encoding = UTF-8 Unicode
\documentclass[aspectratio=169,12pt]{beamer}
\usepackage[utf8]{inputenc}
\usepackage[japanese]{babel}
\usepackage{graphicx}
\usepackage{amsmath}
\usepackage{amssymb}
\usepackage{url}

% テーマとカラーの設定
\usetheme{Madrid}
\usecolortheme{default}

\title{階層的世界モデルの現状と課題:\\Hierosの限界と将来の展望}
\subtitle{2分間研究概要}
\author{三好理輝 \and 劉智優 \and 山田達也}
\institute{ケンブリッジ大学 \and 電気通信大学 \and 大阪大学}
\date{\today}

\begin{document}

% タイトルスライド
\begin{frame}
\titlepage
\end{frame}

% スライド1: 研究背景
\begin{frame}{研究背景}
\begin{columns}
\begin{column}{0.6\textwidth}
\begin{itemize}
\item \textbf{階層的強化学習 (HRL)} 
  \begin{itemize}
  \item 長期計画に有効
  \item 効率的な探索を実現
  \end{itemize}
\item \textbf{世界モデル}
  \begin{itemize}
  \item 高いサンプル効率
  \item 少ない環境相互作用で学習
  \end{itemize}
\item \textbf{統合アプローチ}: Director, Hieros
  \begin{itemize}
  \item ベースライン手法を上回る性能
  \item しかし内部メカニズムは未解明
  \end{itemize}
\end{itemize}
\end{column}
\begin{column}{0.4\textwidth}
\includegraphics[width=\textwidth]{media/videos/report/subgoal_visualization_395000_ea3189286730c79ae915.gif}
\end{column}
\end{columns}
\end{frame}

% スライド2: 研究目的
\begin{frame}{研究目的と手法}
\begin{block}{研究の動機}
階層的世界モデル(Hieros)は本当に「階層的な計画」を学習しているのか?
\end{block}

\begin{columns}
\begin{column}{0.5\textwidth}
\textbf{検証手法}
\begin{itemize}
\item Visual Pinpad環境での評価
\item Atari環境での内部状態可視化
\item ハイパーパラメータ感度分析
\item 階層数の影響調査
\end{itemize}
\end{column}
\begin{column}{0.5\textwidth}
\includegraphics[width=\textwidth]{media/pinpad/Hieros-baseline.png}
\end{column}
\end{columns}
\end{frame}

% スライド3: 実験環境
\begin{frame}{実験環境}
\begin{columns}
\begin{column}{0.5\textwidth}
\textbf{Visual Pinpad}
\begin{itemize}
\item 長期記憶が必要
\item 論理的順序推論
\item 赤→緑→青の順序タスク
\end{itemize}

\vspace{0.5cm}

\textbf{Atari}
\begin{itemize}
\item 反射的行動が中心
\item 高スコア≠高度な計画?
\end{itemize}
\end{column}
\begin{column}{0.5\textwidth}
\includegraphics[width=\textwidth]{media/videos/train_stats/policy_image_368164_32b4f1098e74a1f3aa9c.gif}
\end{column}
\end{columns}
\end{frame}

% スライド4: 主要な発見
\begin{frame}{主要な発見}
\begin{alertblock}{実験結果}
\begin{enumerate}
\item \textbf{Visual Pinpad}: ハイパーパラメータに強く依存、限定的頑健性
\item \textbf{Atari}: 高スコアでも単純な行動パターンのみ
\item \textbf{階層数増加}: 学習安定性が低下
\end{enumerate}
\end{alertblock}

\begin{columns}
\begin{column}{0.5\textwidth}
\includegraphics[width=\textwidth]{media/images/hierarchy_analysis.png}
\end{column}
\begin{column}{0.5\textwidth}
\includegraphics[width=\textwidth]{media/pinpad/reward-design-sweep/sweep-episode-scores.png}
\end{column}
\end{columns}
\end{frame}

% スライド5: 可視化による洞察
\begin{frame}{内部状態可視化による洞察}
\begin{columns}
\begin{column}{0.6\textwidth}
\textbf{サブゴール可視化の結果}
\begin{itemize}
\item 上位層は期待される長期サブゴール(青)を提示せず
\item 短期的サブゴール(赤)のみに留まる  
\item 階層的抽象化が機能していない
\end{itemize}

\textbf{探索ヒートマップ}
\begin{itemize}
\item 探索範囲が限定的
\item エントロピー調整でも根本的改善なし
\end{itemize}
\end{column}
\begin{column}{0.4\textwidth}
\includegraphics[width=\textwidth]{media/videos/report/subgoal_visualization_96096_d09f1268047ab4e15d08.gif}
\end{column}
\end{columns}
\end{frame}

% スライド6: 数式による理解
\begin{frame}{階層的報酬設計の検証}
\textbf{検証した報酬関数}

\begin{align}
r_{\text{env}}(s,a) &= \text{環境からの外部報酬} \\
r_g(s,g) &= \text{サブゴール達成報酬} \\
r_{\text{nov}}(s) &= \beta \cdot \frac{\|\phi(s) - \mu_{\text{buffer}}\|^2}{\sigma_{\text{buffer}}^2 + \epsilon} \\
r_{\text{total}} &= r_{\text{env}} + \lambda_g r_g + \lambda_{\text{nov}} r_{\text{nov}}
\end{align}

\begin{block}{結果}
報酬設計を工夫してもHierosの学習は改善されず、\\
根本的な構造的課題の存在が示唆された
\end{block}
\end{frame}

% スライド7: 結論と将来展望
\begin{frame}{結論と将来展望}
\begin{block}{主要な結論}
現在の階層的世界モデルには根本的課題が存在:
\begin{itemize}
\item 理論的優位性と実際の性能にギャップ
\item ハイパーパラメータ頑健性の欠如
\item 階層化による最適化の複雑性増大
\end{itemize}
\end{block}

\begin{exampleblock}{今後の研究方向}
\begin{enumerate}
\item \textbf{動的階層抽象化}:固定的時間抽象化の改善
\item \textbf{頑健性向上}:ハイパーパラメータ感度の軽減
\item \textbf{理論的発展}:階層化に伴う最適化理論
\end{enumerate}
\end{exampleblock}
\end{frame}

% スライド8: まとめ
\begin{frame}{まとめ}
\begin{center}
\large
階層的世界モデルの現状を包括的に評価し、\\
\textbf{Hierosの構造的限界}を明らかにした。

\vspace{1cm}

今後は\textbf{動的階層化}と\textbf{最適化理論}の発展が\\
階層的強化学習の実用化に不可欠。

\vspace{1cm}

\textbf{ご清聴ありがとうございました}
\end{center}
\end{frame}

\end{document}